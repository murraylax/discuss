\documentclass{beamer}
\usepackage{beamerthemeshadow}
\usepackage{verbatim}

\usepackage{lastpage}
\usepackage{xcolor}
\usepackage{pgf}
\usepackage{colortbl}
\usepackage{hyperref}

\newcommand{\bi}{\begin{itemize}}
\newcommand{\ei}{\end{itemize}}
\newcommand{\be}{\begin{enumerate}}
\newcommand{\ee}{\end{enumerate}}
\newcommand{\bd}{\begin{description}}
\newcommand{\ed}{\end{description}}
\newcommand{\prbf}[1]{\textbf{#1}}
\newcommand{\prit}[1]{\textit{#1}}
\newcommand{\beq}{\begin{equation}}
\newcommand{\eeq}{\end{equation}}
\newcommand{\bdm}{\begin{displaymath}}
\newcommand{\edm}{\end{displaymath}}

\newcommand{\ft}[1]{
  \frametitle{\begin{tabular}{p{4.2in}r} \textcolor{white}{#1} & \small{\insertframenumber / \inserttotalframenumber} \end{tabular}}
  \setbeamercovered{transparent=18}
}

\newcommand{\eft}[1]{
  \frametitle{\begin{tabular}{p{4in}r} \textcolor{white}{#1} & \small{\hyperlink{f:questions}{\beamergotobutton{GO BACK}}} \end{tabular}}
  \setbeamercovered{transparent=18}
}

\newcommand{\stepinv}{\setbeamercovered{invisible}}
\newcommand{\stopinv}{\setbeamercovered{transparent=18}}
\newcommand{\uncoverinv}[1]
{
  \setbeamercovered{invisible}
  \uncover<+->{#1}
  \setbeamercovered{transparent=18}
}
\newcommand{\ans}[1]{\textcolor{blue}{#1}}
\newcommand{\ansinv}[1]
{
  \setbeamercovered{invisible}
  \uncover<+->{\textcolor{blue}{#1}}
  \setbeamercovered{transparent=18}
}
\newcommand{\setinv}{\setbeamercovered{invisible}}
\newcommand{\setvis}{\setbeamercovered{transparent=18}}
\newcommand{\centerpic}[2]
{
  \begin{center}
  \includegraphics[#1]{#2}
  \end{center}
}
\newcommand{\h}[1]{\hat{#1}}
\newcommand{\ds}{\displaystyle}

%\definecolor{light}{rgb}{1.0,0.33,0.33}
\definecolor{light}{rgb}{1.0,0.5,0.5}
\newcommand{\hl}[1]{\alt<#1>{\rowcolor{light}\hspace*{-2.1pt}} {\hspace*{-2.1pt}} }

\definecolor{mycolor}{rgb}{0.6,0.0,0.0}
\usecolortheme[named=mycolor]{structure}

\title[Discussion: Targeted Vaccine for Healthcare Workers]{Discussion: Targeted Vaccine Subsidies for Healthcare Workers}
\subtitle{By T. Tasseir, P. Polgreen, and A. Segre}
\author[James Murray, University of Wisconsin - La Crosse]{James Murray\\Department of Economics\\University of Wisconsin - La Crosse}
\date{November 21, 2009}

\begin{document}

\frame{\titlepage \setcounter{framenumber}{0}}

\frame
{
  \ft{Summary}
  \begin{block}{Susceptible-Infected-Recovered (SIR) Model}
  \bi
  \item<+-> Extended for heterogeneous contact:
  \item<+-> Possible disconnect in Prob(infected) and Marginal Infections.
  \item<+-> Vaccination: low private benefit, but high social benefit.
  \ei
  \end{block}

  \begin{block}{Model Simulation}
  \bi
  \item<+-> Collected contact data on 140 workers in 16 different roles in hospital.
  \item<+-> Used averages to calibrate contact frequency heterogeneity.
  \item<+-> Found simulations for marginal infections, expected marginal infections. 
  \item<+-> Some expected and surprising results extremely relevant for H1N1 policy.
  \item<+-> What is a unit clerk?  
  \ei
  \end{block}
}

\frame
{
  \ft{General Population}
  \bi
  \item<+-> ``General Population'' refers to population of hospital workers / patients.
  \item<+-> Protecting patient population is probably most crucial in protecting general population.
  \item<+-> Patients could be your agent $j$ connecting:
    \bi
    \item<+-> Group A: Hospital employees.
    \item<+-> Group B: Actual general population.
    \item<+-> Possibly have much higher marginal infections (in general population)
    \ei
  \item<+-> Use a term ``Hospital Population'' and tone down its importance.
  \ei
}

\frame
{
  \ft{Statistics}
  \bi
  \item<+-> There are no standard deviations / confidence intervals / hypothesis tests.
  \item<+-> Table 1 shows average contacts between worker categories.
  \item<+-> These means have standard deviations, normal sampling distributions.
  \item<+-> Monte-Carlo simulation:
    \be
    \item<+-> Draw a sample from sampling distributions Table 1.
    \item<+-> Generate results of marginal infections 
    \item<+-> Repeat a couple million times.
    \ee
  \item<+-> Result: means and standard deviations for results in Tables 3,4,6,7.
  \ei
}

\frame
{
  \ft{Transmission Probability}
  \bi
  \item<+-> Transmission probability, $\alpha$, identical across groups and contact types.
  \item<+-> Data was collected on:
    \be
    \item<.-> Length of contact time.
    \item<.-> Whether physical contact was made.
    \item<.-> Whether hand washing / sanitizing was done.
    \ee
  \item<+-> Maybe summarize this data in a Matrix like Table 1.
  \item<+-> Might provide clues:
    \bi
    \item<+-> Is transmission rate independent of group and contact type?
    \item<+-> Is transmission rate lower around patients?
    \item<+-> Are certain groups' transmission rates different around patients?
    \ei
  \ei
}


\end{document}

