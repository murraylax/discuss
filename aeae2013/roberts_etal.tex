\documentclass{beamer}
\usepackage{beamerthemeshadow}

%\documentclass{article}
%\usepackage{beamerarticle}
%\usepackage{graphicx}

\usepackage{verbatim}
%\usepackage{lastpage}
\usepackage{xcolor}
\usepackage{pgf}
\usepackage{colortbl}
\usepackage{hyperref}

\newcommand{\bi}{\begin{itemize}}
\newcommand{\ei}{\end{itemize}}
\newcommand{\be}{\begin{enumerate}}
\newcommand{\ee}{\end{enumerate}}
\newcommand{\bd}{\begin{description}}
\newcommand{\ed}{\end{description}}
\newcommand{\prbf}[1]{\textbf{#1}}
\newcommand{\prit}[1]{\textit{#1}}
\newcommand{\beq}{\begin{equation}}
\newcommand{\eeq}{\end{equation}}
\newcommand{\bdm}{\begin{displaymath}}
\newcommand{\edm}{\end{displaymath}}

\newcommand{\ft}[1]{
  \frametitle{\begin{tabular}{p{4.2in}r} \textcolor{white}{#1} & \small{\insertframenumber / \inserttotalframenumber} \end{tabular}}
  %\frametitle{#1}
  \setbeamercovered{transparent=18}
}

\newcommand{\stepinv}{\setbeamercovered{invisible}}
\newcommand{\stopinv}{\setbeamercovered{transparent=18}}
\newcommand{\uncoverinv}[1]
{
  \setbeamercovered{invisible}
  \uncover<+->{#1}
  \setbeamercovered{transparent=18}
}
\newcommand{\ans}[1]{\textcolor{blue}{#1}}
\newcommand{\ansinv}[1]
{
  \setbeamercovered{invisible}
  \uncover<+->{\textcolor{blue}{#1}}
  \setbeamercovered{transparent=18}
}
\newcommand{\setinv}{\setbeamercovered{invisible}}
\newcommand{\setvis}{\setbeamercovered{transparent=18}}
\newcommand{\centerpic}[2]
{
  \begin{center}
  \includegraphics[#1]{#2}
  \end{center}
}
\newcommand{\h}[1]{\hat{#1}}
\newcommand{\ds}{\displaystyle}

%\definecolor{light}{rgb}{0.17,0.55,0.35}
\newcommand{\hl}[1]{\alt<#1>{\rowcolor{light}\hspace*{-2.1pt}} {\hspace*{-2.1pt}} }

\definecolor{mycolor}{rgb}{0.17,0.55,0.35}
\usecolortheme[named=mycolor]{structure}

\title[Discussion: Roberts, et. al. (2013)]{Discussion of ``Lesson Study: Teaching Economics Life Skills with Student Activities''}
\subtitle{By Helen Roberts, Joy Joyce, Carlos Villarreal, and Luis Guillermo Serpa}
\author[James Murray, University of Wisconsin - La Crosse]{James Murray\\Department of Economics\\University of Wisconsin - La Crosse}
\date{Thursday, May 30, 2013}

\begin{document}

\frame{\titlepage \setcounter{framenumber}{0}}

\frame
{
  \ft{Overall Impressions}
  \bi
  \item Authors document lessons that are part of a very worthwhile cause.
  \item Are these programs and/or lessons ongoing?
  \item Have you collected data on students' learning and preexisting knowledge?
  \item Do improvements in financial literacy improve behavior?  Maybe cite something here.
  \item Did you submit any of these lessons to \textcolor{blue}{\href{http://serc.carleton.edu/econ/}{Starting Point}} (\textcolor{blue}{\href{http://serc.carleton.edu/econ/}{http://serc.carleton.edu/econ/}})?
  \ei
}

\frame
{
  \ft{Collect Evidence on Learning}
  \begin{block}{Motivate it with some data.}
    \bi
    \item Pre-test with students.
    \item Focus group conversations with students on the first day.
    \item Indirect assessment: student survey.
    \ei
  \end{block}

  \begin{block}{Gather Data on Student Learning}
    \bi
    \item Direct assessment measure: rubrics aligned with goals of the lessons.
    \item Classroom observations using a classroom observation guide.
    \item Indirect assessment: post-lesson survey
      \bi
      \item Open ended: What was the most interesting thing you learned in the lesson?
      \item Students could evaluate their own learning on the goals.
      \item You could ask students about their own plans, behaviors.
      \ei
    \ei
  \end{block}
}

\frame
{
  \ft{What to look for in the data?}
 
    \bi
    \item Is this lesson ``effective?''  I don't care.
    \item With what concepts did students demonstrate the most growth?
    \item What concepts remain challenging?
    \ei

}

\end{document}

