\documentclass{beamer}
\usepackage{beamerthemeshadow}
\usepackage{verbatim}

\usepackage{lastpage}
\usepackage{xcolor}
\usepackage{pgf}
\usepackage{colortbl}
\usepackage{hyperref}

\newcommand{\bi}{\begin{itemize}}
\newcommand{\ei}{\end{itemize}}
\newcommand{\be}{\begin{enumerate}}
\newcommand{\ee}{\end{enumerate}}
\newcommand{\bd}{\begin{description}}
\newcommand{\ed}{\end{description}}
\newcommand{\prbf}[1]{\textbf{#1}}
\newcommand{\prit}[1]{\textit{#1}}
\newcommand{\beq}{\begin{equation}}
\newcommand{\eeq}{\end{equation}}
\newcommand{\bdm}{\begin{displaymath}}
\newcommand{\edm}{\end{displaymath}}

\newcommand{\ft}[1]{
  \frametitle{\begin{tabular}{p{4.2in}r} \textcolor{white}{#1} & \small{\insertframenumber / \inserttotalframenumber} \end{tabular}}
  \setbeamercovered{transparent=18}
}

\newcommand{\eft}[1]{
  \frametitle{\begin{tabular}{p{4in}r} \textcolor{white}{#1} & \small{\hyperlink{f:questions}{\beamergotobutton{GO BACK}}} \end{tabular}}
  \setbeamercovered{transparent=18}
}

\newcommand{\stepinv}{\setbeamercovered{invisible}}
\newcommand{\stopinv}{\setbeamercovered{transparent=18}}
\newcommand{\uncoverinv}[1]
{
  \setbeamercovered{invisible}
  \uncover<+->{#1}
  \setbeamercovered{transparent=18}
}
\newcommand{\ans}[1]{\textcolor{blue}{#1}}
\newcommand{\ansinv}[1]
{
  \setbeamercovered{invisible}
  \uncover<+->{\textcolor{blue}{#1}}
  \setbeamercovered{transparent=18}
}
\newcommand{\setinv}{\setbeamercovered{invisible}}
\newcommand{\setvis}{\setbeamercovered{transparent=18}}
\newcommand{\centerpic}[2]
{
  \begin{center}
  \includegraphics[#1]{#2}
  \end{center}
}
\newcommand{\h}[1]{\hat{#1}}
\newcommand{\ds}{\displaystyle}

%\definecolor{light}{rgb}{1.0,0.33,0.33}
\definecolor{light}{rgb}{1.0,0.5,0.5}
\newcommand{\hl}[1]{\alt<#1>{\rowcolor{light}\hspace*{-2.1pt}} {\hspace*{-2.1pt}} }

\definecolor{mycolor}{rgb}{0.6,0.0,0.0}
\usecolortheme[named=mycolor]{structure}

\title[Discussion of Escobari: Stochastic Peak-Load Pricing]{WEAI Discussion of Escobari (2010):}
\subtitle{Stochastic Peak-Load Pricing and Aggregate Demand Learning in Airlines}
\author[James Murray, University of Wisconsin - La Crosse]{James Murray\\Department of Economics\\University of Wisconsin - La Crosse}
\date{July 1, 2010}

\begin{document}

\frame{\titlepage \setcounter{framenumber}{0}}

\frame
{
  \ft{Compliments}
  \bi
  \item Provide a great overview of the literature that motivates the present work.
  \item Explain well the complexity surrounding airline pricing.
  \item Econometric methodology:
    \bi
    \item Seems to be well motivated by literature - current practices.
    \item I like the threshold model.
    \item Identify a signal, $S_{ijt}$, that informs whether a flight is likely to be ``peak'' or ``off-peak''.
    \item Take care to account for endogeneity in days in advance, and load.
    \ei
  \item Paper seems to be well polished, ready for submission.
  \ei
}

\frame
{
  \ft{Figure 2: Bottom Panel}
  \begin{columns}
    \column{1.5in}
    \bi
    \item No ``booking curve'' yet.
    \item It's not obvious what this should be.
    \item Keep the figure, but admit it isn't obvious.  This motivates your work.
    \ei
    \column{2.5in}
    \centerpic{scale=0.2}{diego.png}
  \end{columns}
}

\frame
{
  \ft{Critical Reactions}
  \bi
  \item Econometric modeling:
    \bi
    \item Why does the signal only impact the coefficient on load?
    \item What dynamics are you explaining over-and-above a simple model?
    \item Is there a priori evidence of switching coefficients in these regressions?
    \ei
  \item Why is this learning?
    \bi
    \item With learning, uncertainty decreases as time progresses (flight time approaches).
    \item With learning, expectations (booking curve?) react to past signals.
    \item You might capture some of these dynamics with interaction of signal on load.
    \ei
  \ei
}

\frame
{
  \ft{Final reactions}
  \bi
  \item Motivation for someone not familiar with the literature is not convincing.
  \item I don't get a sense as to whether airlines are behaving optimally.
  \item Statistical evidence that signal causes switches in coefficients on price discrimination characteristics?
  \ei
}

\end{document}

