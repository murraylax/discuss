\documentclass{beamer}
\usepackage{beamerthemeshadow}

%\documentclass{article}
%\usepackage{beamerarticle}
%\usepackage{graphicx}

\usepackage{verbatim}
%\usepackage{lastpage}
\usepackage{xcolor}
\usepackage{pgf}
\usepackage{colortbl}

\newcommand{\bi}{\begin{itemize}}
\newcommand{\ei}{\end{itemize}}
\newcommand{\be}{\begin{enumerate}}
\newcommand{\ee}{\end{enumerate}}
\newcommand{\bd}{\begin{description}}
\newcommand{\ed}{\end{description}}
\newcommand{\prbf}[1]{\textbf{#1}}
\newcommand{\prit}[1]{\textit{#1}}
\newcommand{\beq}{\begin{equation}}
\newcommand{\eeq}{\end{equation}}
\newcommand{\bdm}{\begin{displaymath}}
\newcommand{\edm}{\end{displaymath}}

\newcommand{\ft}[1]{
  \frametitle{\begin{tabular}{p{4.2in}r} \textcolor{white}{#1} & \small{\insertframenumber / \inserttotalframenumber} \end{tabular}}
  %\frametitle{#1}
  \setbeamercovered{transparent=18}
}

\newcommand{\stepinv}{\setbeamercovered{invisible}}
\newcommand{\stopinv}{\setbeamercovered{transparent=18}}
\newcommand{\uncoverinv}[1]
{
  \setbeamercovered{invisible}
  \uncover<+->{#1}
  \setbeamercovered{transparent=18}
}
\newcommand{\ans}[1]{\textcolor{blue}{#1}}
\newcommand{\ansinv}[1]
{
  \setbeamercovered{invisible}
  \uncover<+->{\textcolor{blue}{#1}}
  \setbeamercovered{transparent=18}
}
\newcommand{\setinv}{\setbeamercovered{invisible}}
\newcommand{\setvis}{\setbeamercovered{transparent=18}}
\newcommand{\centerpic}[2]
{
  \begin{center}
  \includegraphics[#1]{#2}
  \end{center}
}
\newcommand{\h}[1]{\hat{#1}}
\newcommand{\ds}{\displaystyle}

%\definecolor{light}{rgb}{0.17,0.55,0.35}
\newcommand{\hl}[1]{\alt<#1>{\rowcolor{light}\hspace*{-2.1pt}} {\hspace*{-2.1pt}} }

\definecolor{mycolor}{rgb}{0.17,0.55,0.35}
\usecolortheme[named=mycolor]{structure}

\title[Discussion: Plante and Traum]{Discussion of ``Time-varying Oil Price Volatility and Macroeconomic Aggregates: What Does Theory Say''}
\subtitle{By Michael Plante and Nora Traum}
\author[James Murray, University of Wisconsin - La Crosse]{James Murray\\Department of Economics\\University of Wisconsin - La Crosse}
\date{Friday, March 18, 2011}

\begin{document}

\frame{\titlepage \setcounter{framenumber}{0}}

\frame
{
  \ft{Summary}
\bi
\item Purpose: Expose the responses of consumer and producer decisions to an increase in oil price volatility.
\item Put oil in the production function, discuss the effect of elasticity of complementarity/substitution for oil and other inputs in production.
\item Put oil in the utility function along with durable goods, discuss the effect of elasticity of substitution between oil and durables.
\item Reasonable calibrations: 
  \bi
  \item Investment and output rise in response to higher uncertainty.
  \item Output falls only when focusing on durables (no capital).
  \ei
\ei
}

\frame
{
  \ft{Saving Supply Channel}
  \bi
  \item Increase in oil price uncertainty $\rightarrow$ increase in income uncertainty.
  \item Risk averse consumers:
    \bi
    \item Increase precautionary savings (increase investment)
    \item Decrease consumption
    \item Increase labor?
    \ei
  \item Counter-intuitive result: increase in investment and output.
  \ei
}

\frame
{
  \ft{Investment Demand Channel}
  \bi
  \item Not discussed in the paper, but it is evident from FOCs.
  \item Increase in oil price uncertainty $\rightarrow$ increase in $MP_K$ uncertainty.
  \item This increases uncertainty about future income stream ($r_{t+1} K_{t+1}$) from renting capital.
  \item Risk averse behavior = decrease in investment.
  \ei
}

\frame
{
  \ft{Flock to Capital in Times of Uncertainty}
  \bi
  \item Saving supply channel outweighs investment demand channel.
  \item Model: little inherent risk to purchasing capital.
    \bi
    \item Only uncertainty about $r_{t+1}$.
    \item Capital can still be costlessly be converted to consumption.
    \ei
  \item Make investment in capital risky.
  \ei
}

\frame
{
  \ft{Irreversible Capital}
  \begin{block}{Investment is Risky}
    \bi
    \item Capital can only be imperfectly (or not at all) converted to consumption 
    \item Diminishes saving supply channel.
    \item Investment demand falls, output falls.
    \ei
  \end{block}

  \begin{block}{Literature}
  \bi
  \item Theory: Bernanke (QJE, 1983)
  \item Statistical and economic significance: Chirinko and Schaller (JME, 2009)
  \item Endogenous time-varying volatility of oil prices and irreversible investment in \textit{oil production}: Kogan, Livdan, and Yargon (Journal of Finance, 2009)
  \ei
  \end{block}
}

\frame
{
  \ft{Signal Extraction?}
  \bi
  \item Paper assumes a shock to the volatility of a shock is perfectly observable.
  \item Hamilton's (1996) point?  
    \bi
    \item Decreases in oil prices were often ``market corrections.''  
    \item These became signals of an increase in volatility.
    \ei
  \item Expectations constructed as they are, don't expect to match shape of empirical IRFs.
  \item Perhaps this is another paper.
  \ei
}

\end{document}

