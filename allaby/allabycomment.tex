\documentclass[12pt]{article}
\usepackage[T1]{fontenc}
\usepackage{calc}
\usepackage{setspace}
\usepackage{multicol}

\usepackage{graphicx}
\usepackage{color}
\usepackage{rotating}
\usepackage{harvard}
\usepackage{aer}
\usepackage{aertt}
\usepackage{verbatim}

\setlength{\voffset}{0in}
\setlength{\topmargin}{0pt}
\setlength{\hoffset}{0pt}
\setlength{\oddsidemargin}{0pt}
\setlength{\headheight}{0pt}
\setlength{\headsep}{.1in}
\setlength{\marginparsep}{0pt}
\setlength{\marginparwidth}{0pt}
\setlength{\marginparpush}{0pt}
\setlength{\footskip}{.4in}
\setlength{\textwidth}{6.5in}
\setlength{\textheight}{9in}
\setlength{\parskip}{0pc}

\renewcommand{\baselinestretch}{1.5}

\newcommand{\bi}{\begin{itemize}}
\newcommand{\ei}{\end{itemize}}
\newcommand{\be}{\begin{enumerate}}
\newcommand{\ee}{\end{enumerate}}
\newcommand{\bd}{\begin{description}}
\newcommand{\ed}{\end{description}}
\newcommand{\prbf}[1]{\textbf{#1}}
\newcommand{\prit}[1]{\textit{#1}}
\newcommand{\beq}{\begin{equation}}
\newcommand{\eeq}{\end{equation}}
\newcommand{\bdm}{\begin{displaymath}}
\newcommand{\edm}{\end{displaymath}}
\newcommand{\script}[1]{\begin{cal}#1\end{cal}}
\newcommand{\citee}[1]{\citename{#1} (\citeyear{#1})}

\newcommand{\h}[1]{\hat{#1}}
\newcommand{\ds}{\displaystyle}

\newcommand{\app}
{
\appendix
\renewcommand{\theequation}{A\arabic{equation}}
\setcounter{equation}{0}
}

\newcommand{\appsection}[1]
{
\let\oldthesection\thesection
\renewcommand{\thesection}{Appendix \oldthesection}
\section{#1}\let\thesection\oldthesection
}

\begin{document}

\begin{singlespace}
\title{Discussion of ``Feasibility of Corn Ethanol from a Land Use Perspective'' by Andrew Allaby\footnote{Prepared for the Indiana University Economics Department Jordan River Conference 2007.}}
\author{James Murray\\
Department of Economics\\
Indiana University\footnote{Mailing address: 100 S Woodlawn, Bloomington, IN  47405. E-mail: jmmurray@indiana.edu.  Phone: (574)315-0459.}}
\date{April 20, 2007}
\maketitle
\end{singlespace}

Motivated by President George W. Bush's stated ambition that U.S. production of ethanol increase to 35 billion gallons by the year 2016, Allaby examines evidence of feasible increases in ethanol production.  He examines improvements in ethanol production stemming from three sources: improvements in corn crop yield, improvements in ethanol yield per bushel of corn, and increases in land acreage dedicated to corn, and specifically ethanol production.  Allaby suggests how new technologies for growing high starch variants of corn, specific for ethanol production, and how new technologies for distilling ethanol might increase ethanol yield.  On the corn crop yield source, Allaby supposes over the next ten years, crop yield will continue to grow at the same linear rate as has occurred since the 1940s.  Finally Allaby concentrates primarily on how farming acreage for corn may increase over the next ten years.  He cites figures on how many acres of competing crops may be replaced with corn production and how many acres may be released from the Conservation Reserve Program and used for corn production.

Allaby conducts numerical exercises to make low, medium, and high projections for U.S. ethanol output in 2016.  These are based on continuing linear trends in crop yield and ethanol yield, statistics cited from other sources for added acreage to corn production, and assuming that per bushel corn prices remain fixed.  All of these are clearly unrealistic assumptions and can be very damaging to forecast accuracy when attempting to forecast 10 years out of sample.  An especially unrealistic assumption is that corn prices remain fixed.  The premise of the paper is that somehow domestic demand for ethanol should drastically increase over the next ten years, due to perhaps continuing increases in fossil fuel prices, or environmentally motivated changes in federal taxes on fossil fuels or federal subsidies on bio-fuels.  The paper does not indicate a reason for an increase in ethanol demand.  Given an increase in demand does occur, one should expect ethanol prices, and therefore corn prices to rise.  The increase in the return on ethanol will motivate farmers to increase acreage dedicated to corn production and motivate them to develop technologies to increase ethanol yield.  These economics considerations are explicitly ignored in the paper.  

Allaby's paper gives an interesting description as to how ethanol production can be increased in the United States, but his forecasts to the year 2016 should not be taken seriously.  Also missing form the paper is a economic description of how demand for ethanol should increase over ten years, how federal programs will encourage ethanol production, and how farmers' choices will respond to changes corn prices.  An improved paper will discuss these issues, and leave out the numerical forecasts.

\end{document}



